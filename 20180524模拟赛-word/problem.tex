\documentclass[hyperref,UTF8,12pt,a4paper]{ctexart}

\usepackage{amsmath,multirow,makecell,algorithm}

\usepackage{array}
\newcolumntype{P}[1]{>{\centering\arraybackslash}p{#1}}

\usepackage{geometry}
\geometry{left=1in,right=1in,top=1in,bottom=1in}

\usepackage{titling}
\pretitle{\begin{center}\fontsize{30pt}{30pt}\selectfont}
\posttitle{\end{center}}

\usepackage{algpseudocode, graphicx}

\title{Word}
\author{}
\date{}

\begin{document}

\section{Word}

\subsection{题目描述}

有一个星球要创造新的单词,单词有一些条件:

\begin{itemize}
\item [1)]
字母集有$p$个元音和$q$个辅音,单词由字母构成。
\item [2)]
每个单词最多有$n$个元音和$n$个辅音(同一元音或辅音可重复使用)。
\item [3)]
每个单词中元音总是出现在所有辅音之前,可以没有元音或没有辅音,这里的“所有”指的是“全部”,即如果`a`是元音,`b`是辅音,那么`aabb`是合法的,而`abab`是不合法的。
\item [4)]
每个单词至少有一个字母。
\item [5)]
可以在字母上标记重音。元音中最多标记一个,辅音中也最多标记一个,一个单词中最多标记两个字母为重音。
\item [6)]
如果两个单词字母、字母顺序或者重音不同就认为这两个单词不同。
\end{itemize}

他们想要知道一共能创造多少不同的单词,由于答案可能很大,所以只要输出答案`mod M`后的值。

\subsection{输入输出格式}

\subsubsection{输入格式}

输入文件`word.in`包含4个正整数$p,q,n,M$

\subsubsection{输出格式}

输出文件`word.out`包含一个非负整数表示能创造出的新单词`mod M`后的值。

\subsection{样例}

\subsubsection{样例输入1}

\begin{verbatim}
1 1 1 9
\end{verbatim}

\subsubsection{样例输出1}

\begin{verbatim}
8
\end{verbatim}

\subsubsection{样例输入2}

\begin{verbatim}
2 3 2 1000
\end{verbatim}

\subsubsection{样例输出2}

\begin{verbatim}
577
\end{verbatim}

\subsubsection{样例输入3}

\begin{verbatim}
1 1 1000000000 1000000000 
\end{verbatim}

\subsubsection{样例输出3}

\begin{verbatim}
0 
\end{verbatim}

\subsection{数据规模}

对于 $30\%$的数据,$p, q, n\le 7$

对于$60\%$的数据,$n \le 100000$

对于$100\%$的数据,$p, q, n, M ≤ 10^9$

\end{document}
